\chapter{INTRODUÇÃO}

O número de aplicações para a internet tem crescido radicalmente nos ultimos anos, mais especificamente, desde que a internet deixou de ser composta majoritariamente por conteúdo estático e permitiu a interação do usuário com seus conteúdos. Desde o aparecimento da Web 2.0, o usuário vem conquistando progressivamente mais controle sobre aplicações para internet. Isso pode ser ilustrado pelo engrandecimento do Google e seu motor de busca, e a desvaliação do Yahoo! e seu catálogo de páginas.

\textbf{To be continued.}

%Porém, para que as aplicações possam, de cada vez mais, oferecer mais controle para seus usuários

% o qual permite que o usuário digite palavras-chave para criar uma busca personalizada.


% O objetivo deste documento é apresentar as tarefas realizadas e as decisões tomadas durante o desenvolvimento do módulo de análise do Sistema de Outorga do estado do Rio Grande do Sul -- SIOUT RS -- tendo um enfoque nas estratégias adotadas para a construção da interface.

% O sistema descrito neste documento foi desenvolvido em duas partes: A primeira delas foi a interface gráfica, a qual faz uso dos frameworks: Bootstrap -- um framework que contém componentes HTML e CSS para agilizar o desenvolvimento de websites e sistemas web -- e AngularJS -- um framework Javascript que utiliza a arquitetura MVW \cite{branas2014angularjs} para auxiliar o desenvolvimento de sistemas complexos.