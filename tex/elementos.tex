\chapter{ELEMENTOS DO TEXTO}
\label{cap:elementos}

Este capítulo apresenta o uso básico de equações, figuras e tabelas no código da monografia, bem como o posicionamento das legendas, segundo as normas da UFLA.

\section{Utilizando Recursos do \LaTeX}

\subsection{Inserindo Comandos Definidos}

Esta subseção apresenta o uso de comandos definidos pelo usuário no preâmbulo do arquivo principal \LaTeX e alguns exemplos do modo matemático. Por exemplo, na texto a seguir é utilizado o comando  \verb+\defs+, definido anteriormente pelo próprio autor do texto:

\begin{quote}
Os conjuntos fundamentais da teoria são os \defs{conjuntos elementares}. Se $E$ é um conjunto elementar, $des(E)$ denota a descrição dessa classe de equivalência. Essa descrição é função do conjunto de atributos que define $R$. Note que, dados $x,y \in E$, onde $E$ é um conjunto elementar em $A$, $x$ e $y$ são indiscerníveis, i.e., no espaço $A=(U,R)$ não se consegue distinguir $x$ de $y$, pois $des(x) = des(y) = des(E)$. 
\end{quote}

\subsection{Inserindo Figuras}

A Figura~\ref{fig:exemplo} é apenas um exemplo de figura para que o usuário da classe possa saber como uma figura pode ser inserida e referenciada automaticamente no texto.É importante observar que legendas de figuras ficam abaixo de seu conteúdo.

\begin{figure}[!htb]
\centering
\caption{Uma Figura de Exemplo} %legenda
\includegraphics[scale=0.9]{images/gradpenguin}\\  % o 0.9 indica 90% do tamanho original
% pdfLaTeX aceita figuras no formato PNG, JPG ou PDF
% figuras vetoriais podem ser exportadas para eps e depois convertidas para pdf usando epstopdf
{\small Fonte: fonte da figura} %Fonte da imagem
\label{fig:exemplo} %rotulo para refencia
\end{figure}

\subsection{Inserindo Saídas de Comandos e Código}

A menos que sejam trechos pequenos, saídas de comandos, trechos de arquivos de configuração e código de aplicativos devem ser inseridos como figura, como mostrado, respectivamente, na Figura~\ref{fig:exemplocomando}, Figura~\ref{fig:exemploconfig} e Figura~\ref{fig:exemplocodigo1}. Para comandos e configuração, recomenda-se o uso do pacote {\tt fancyvrb}, o que pode ser visto na Figura~\ref{fig:exemplocomando} e Figura~\ref{fig:exemploconfig}.

Para inserção de código, recomenda-se o uso do pacote {\tt listings}, que permite melhor apresentação do mesmo, o que é mostrado na Figura~\ref{fig:exemplocodigo1}. Além disso, esses dois pacotes permitem a inserção de texto/código em arquivos externos, sem inclusão direta no arquivo \LaTeX. Isso pode ser verificado no exemplo de uso do {\tt listings} apresentado na Figura~\ref{fig:exemplocodigo2}

\begin{figure}[!htb]
\centering
\caption{Inserindo Comando} %legenda
\begin{Verbatim}[fontsize=\small]
$ dir monografia*
-rw-r--r--  1 joukim users   3650 Set 12 17:56 monografia.aux
-rw-r--r--  1 joukim users   6366 Set 12 17:43 monografia.bbl
-rw-r--r--  1 joukim users    290 Set 12 17:56 monografia.lof
-rw-r--r--  1 joukim users  27937 Set 12 17:56 monografia.log
-rw-r--r--  1 joukim users    194 Set 12 17:56 monografia.lot
-rw-r--r--  1 joukim users    715 Set 12 17:56 monografia.out
-rw-r--r--  1 joukim users 159243 Set 12 17:56 monografia.pdf
-rw-r--r--  1 joukim users   4559 Set 12 17:47 monografia.tex
-rw-r--r--  1 joukim users    964 Set 12 17:56 monografia.toc
\end{Verbatim} 
%$ - esse comentário é para não confundir editor de texto
{\small Fonte: fonte da figura} %Fonte da imagem
\label{fig:exemplocomando} %rotulo para refencia
\end{figure}

\begin{figure}[!htb]
\centering
\caption{Inserindo Trecho de Arquivo de Configuração} %legenda
\begin{Verbatim}[fontsize=\small]
// named.conf for Red Hat caching-nameserver
options {
        directory "/var/named";
        dump-file "/var/named/data/cache_dump.db";
        statistics-file "/var/named/data/named_stats.txt";
        // query-source address * port 53;
};
\end{Verbatim} 
{\small Fonte: fonte da figura} %Fonte da imagem
\label{fig:exemploconfig} %rotulo para refencia
\end{figure}


\begin{figure}[!htb]
\centering
\caption{Inserindo Código Diretamente no Arquivo \LaTeX} %legenda
\begin{lstlisting}
// exit the program
public void on_buttonExit_clicked() {
	System.exit(0);
}

// copy data
public void on_buttonCopy_clicked() {
	labelShow.setText(entryData.getText());
}

// print version of Java
public static void main(String[] args) {
	System.out.println(System.getProperty("java.fullversion"));
}
\end{lstlisting} 
{\small Fonte: fonte da figura} %Fonte da imagem
\label{fig:exemplocodigo1} %rotulo para refencia
\end{figure}


\begin{figure}[H]
\centering
\caption{Inserindo Código a Partir do Código-Fonte} %legenda
\label{fig:exemplocodigo2} %rotulo para refencia
{\small Fonte: fonte da figura} %Fonte da imagem
\end{figure}

\subsection{Inserindo Quadros e Tabelas}

Escrever um quadro ou tabela e referenciá-los é bem simples. Por exemplo o Quadro~\ref{tab:exemplo} ilustra a criação de um quadro, tendo aqui seu referenciamento no texto. É importante observar o posicionamento da legenda antes do corpo da tabela e da fonte após. Outros exemplos são mostrados na Tabela~\ref{tab:outro} e Tabela~\ref{tab:maisum}.

\begin{quadro}[htb]
  \begin{center}
    \caption{Exemplo de Quadro} 
    \label{tab:exemplo}
    \vspace{0.2cm}
    \footnotesize
    \begin{tabular}{|c|c|c|c|c|c|}
      \hline
      $U$ & $vitA$ & $vitC$ & $vitD$ & $prot$ & $lip$ \\
      \hline
      \hline
      $d_1$ & 1 & 3 & 4 & 2 & 3\\
      $d_2$ & 1 & 3 & 3 & 3 & 2\\
      $d_3$ & 1 & 3 & 4 & 3 & 1\\
      $d_4$ & 3 & 5 & 2 & 5 & 2\\
      $d_5$ & 4 & 5 & 2 & 5 & 1\\
      $d_6$ & 3 & 5 & 2 & 3 & 4\\
      $d_7$ & 4 & 4 & 1 & 3 & 2\\
      \hline 
    \end{tabular}
  \end{center}
  \centering {\small Fonte: fonte do quadro} %Fonte do quadro
\end{quadro}

\begin{table}[!htb]
\begin{center}
  \caption{Recursos do {\ttfamily syslog}}
  \label{tab:outro}
  \small
  \begin{tabular}{l|p{9cm}}
    \hline
    \rowcolor[gray]{.9}
    \bf Recurso & \bf {\em Daemons} Associados (Alguns Exemplos) \\
    \hline
    \hline
    \tt kern & \em kernel  \\
    \tt user & processos dos usuários ({\tt ntpd}) \\
    \tt mail & softwares relacionados com o correio eletrônico ({\tt sendmail})\\
    \tt daemon & {\em daemons} do sistema ({\tt gated}, {\tt inetd}, 
    {\tt named}, {\tt ntpd})\\
    \tt auth &  comandos relacionados à autorização e segurança 
    ({\tt login}, {\tt rlogin}, {\tt su}, {\tt passwd}) \\
    \tt lpr & spool de impressão ({\tt lpd})\\
    \tt news & sistema de notícias da usenet ({\tt nnrpd})\\
    \tt uucp & destinado ao {\tt uucp}\\
    \tt cron & relacionado ao {\em daemon} {\tt cron}\\
    \tt mark &  registros de data/hora gerados a intervalos regulares 
    ({\tt syslogd})\\
    \tt local0-7 & 8 tipos de mensagens locais \newline
    ({\tt tcpd -- local7}, {\tt sudo -- local2}, {\tt popper - local0}) \\
    \tt syslog &  mensagens internas ao {\tt syslog}\\
    \tt authpriv & mensagens privadas de autorização\\
    \tt ftp & associado ao {\tt ftpd} ({\em daemon} do {\tt ftp}) \\
    \tt * &  todos os recursos com exceção do {\tt mark}\\
    \hline
  \end{tabular}
\end{center}
\centering {\small Fonte: fonte da tabela} %Fonte da tabela
\end{table}

\begin{table}[!htb]
\caption{Opções do Comando {\ttfamily at}}
\label{tab:maisum}
\begin{center}
  \small
  \begin{tabular}{l|p{9cm}}
    \hline 
    \rowcolor[gray]{.9}
    \bf Opção & \bf Descrição\\
    \hline
    \hline 
    \tt -c & exibe os jobs registrados\\
    \tt -d & remove um job específico\\
    \tt -f & permite que os comandos sejam lidos a partir de um arquivo (não pela
    entrada-padrão)\\
    \tt -l & lista todos os jobs associados a um usuário\\
    \tt -m & envia um e-mail ao usuário quando o job for finalizado\\
    \hline \end{tabular}\\
\end{center}
\centering {\small Fonte: fonte da tabela} %Fonte da tabela
\end{table}

\subsection{Inserindo Equações}

Equações devem ser numeradas, com a numeração, em parênteses à direita da mesma. Isso é ilustrado na Equação~\ref{eq:exemplo}.

\begin{equation}
\label{eq:exemplo}
f'(x) = \int^{x^2}_{x^{-1}} xdx 
\end{equation}


\section{Usando Referências}
A equipe do curso não impõe normas rígidas para o formato a ser adotado nas referências bibliográficas, desde que seja usado um padrão acadêmico conhecido. Caso os autores não possuam um padrão preferencial, recomenda-se o formato estipulado pela ABNT \cite{NBR6023:2002}. A Biblioteca Central da UFLA disponibiliza um manual \cite{BIBUFLA2001} que orienta o uso dessas normas. Se os autores estiverem utilizando \LaTeX, recomenda-se o uso do pacote Abn\TeX\footnote{Disponível em \url{http://abntex.codigolivre.org.br/}.} \cite{Weber2003}. 

Obviamente, recomenda-se a leitura de textos sobre a escrita acadêmica e produção de trabalhos de conclusão para garantir não só qualidade estética e de formatação, mas também de conteúdo. Entre outros, pode-se recomendar a leitura de \cite{Silva2005}, \cite{Martins2000}, \cite{Gil2002}, \cite{Franca2001}, \cite{Eco1996}, \cite{Moura1998}, \cite{Booth2000}, \cite{Hexsel2004}, \cite{Porto2002}, \cite{esmin2005hybrid} e \cite{Henz2003}. 

